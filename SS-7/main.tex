\documentclass[16pt, a4paper]{article}
\usepackage[UTF8]{ctex}
%\usepackage[utf8]{inputenc}
\usepackage[colorlinks,linkcolor=blue]{hyperref}
\usepackage{polyglossia}
\usepackage{fontspec}
\setmainfont{Source Han Serif TC}
\setCJKmainfont{Source Han Serif TC}
\setCJKmonofont{Source Han Serif TC}
\setCJKsansfont{Source Han Serif TC}
\makeindex

\begin{document}
	\title{SS-7/SFNIP: Symbol and Finimed Network Interactive Protocol}
	\author{DiannaoJun}
	\date{NT.5-7}
	\maketitle              % 标题页
	\tableofcontents
	\addcontentsline{toc}{section}{目錄}
		\subsection{檔案信息}
			\begin{table}[ht]\centering
				\begin{tabular}{|c|c|c|}
    				\hline
    					\textbf{檔案編號} & \textbf{檔案別名} & \textbf{檔案更新時間 }\\
					\hline
						SS-7 & SFNIP & \today{} \\
    					SDP-1 & ~ & ~ \\
    				\hline
    				\hline
    					\textbf{檔案替代} & \textbf{檔案引用} & \textbf{檔案狀態} \\
    				\hline
    					SS-7/SDP-1/SINIP & SS-5 & 編寫 \\
    				\hline
    				\hline
    					\multicolumn{2}{|c|}{\textbf{檔案版權}} & \textbf{編訂部門}\\
    				\hline
    					\multicolumn{2}{|c|}{Copyright \textcopyright{} 思博開發團隊 Symbol DT. 2019-2024} & 標準編訂組 \\
    				\hline
    			\end{tabular}
    		\caption{檔案信息}\label{tab:multicolumn}\end{table}
	\section{引言}
		\subsection{前言}
			思博·飛慕網路交換協議(Symbol and Finimed Network Interactive Protocol, SFNIP)是一種利用HTTP/HTTPS協議
			(下文簡稱爲「HTTP」)進行網絡操作格式化數據交互傳輸的規則。是一種在HTTP框架內進行非Web數據傳輸,並允許使用者根據需要進行可選是
			部分的狀態化或持久化的協議。
		\subsection{限定詞語的使用}
			見引用文章
		\subsection{術語}
			\subparagraph{請求與響應} 本條符合「HTTP伺服器-客戶機模型」所述;
			\subparagraph{資源} 一種可以被傳遞的網路數據對象;
			\subparagraph{客戶端} 主動發送請求的程序;
			\subparagraph{服務端} 被動接收請求並響應資源與提供服務的程序;
			\subparagraph{持久化(對於可持久化)} 對於資源的復利用,從而提高時間與傳輸效率或爲某些特定功能提供技術根本;
			\subparagraph{緩存系統(對於可持久化)} 是消息本地存儲的子系統,一種用於控制資源與其他數據存儲、獲取與刪除,以在時間與空間效率上尋求平衡的系統;
			\subparagraph{年齡(對於可持久化)} 一個消息從被獲取到現在的時間;
			\subparagraph{壽命(對於可持久化)} 一個消息從被獲取到過期的時間,即數據的有效期;
			\subparagraph{新鮮的(對於可持久化)} 消息在壽命之內,稱爲「新鮮的」;
			\subparagraph{過期的(對於可持久化)} 消息在壽命之外,稱爲「過期的」。
	\section{格式}
		\subsection{參數}
			\subparagraph{SFNIP版本} 位於報頭內,鍵爲「s\_version」。\textbf{應當}由客戶機指定,否則默認值爲1.1。用以識別軟體協議版本,保證兼容性。
			同時,伺服器返回版本號\textbf{不得}高於客戶機版本;
			\subparagraph{軟體包標識符} 位於報頭內,鍵爲「s\_appacks\_id」。用以識別具體軟體類型,使伺服器依此判斷是否有非法類型的軟體訪問。
				注意,此項\textbf{應當}爲可視ASCII字符,且判斷時忽略大小寫的區別;
		\subsection{請求與響應格式}
			遵循HTTP格式,詳見具體實現。
	\section{實現I——SFNIP-FFW協議}
		\subsection{參數}
			\subparagraph{SFNIP版本} 使用1.1版本的SFNIP協議。
			\subparagraph{軟體包標識符} 軟件包爲「feynco\_finimed\_workspace」。
\end{document}
