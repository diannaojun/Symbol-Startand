\documentclass[16pt, a4paper]{article}
\usepackage[UTF8]{ctex}
%\usepackage[utf8]{inputenc}
\usepackage[colorlinks,linkcolor=blue]{hyperref}
\usepackage{polyglossia}
\usepackage{fontspec}
\setmainfont{Source Han Serif TC}
\setCJKmainfont{Source Han Serif TC}
\setCJKmonofont{Source Han Serif TC}
\setCJKsansfont{Source Han Serif TC}
\makeindex

\begin{document}
	\title{SS-5: Définition de Vocabulaire et Délimitation d'Utilisation}
	\author{DiannaoJun}
	\date{NT.5-7}
	\maketitle              % 标题页
	\tableofcontents
	\addcontentsline{toc}{section}{目錄}
		\subsection{檔案信息}
			\begin{table}[ht]\centering
				\begin{tabular}{|c|c|c|}
    				\hline
    					\textbf{檔案編號} & \textbf{檔案別名} & \textbf{檔案更新時間 }\\
					\hline
						SS-5 & 詞彙釋義及使用限定 & \today{} \\
    				\hline
    				\hline
    					\textbf{檔案替代} & \textbf{檔案引用} & \textbf{檔案狀態} \\
    				\hline
    					SS-5/術語與表示 & ~ & 編寫 \\
    				\hline
    				\hline
    					\multicolumn{2}{|c|}{\textbf{檔案版權}} & \textbf{編訂部門}\\
    				\hline
    					\multicolumn{2}{|c|}{Copyright \textcopyright{} 思博開發團隊 Symbol DT. 2019-2024} & 標準編訂組 \\
    				\hline
    			\end{tabular}
    		\caption{檔案信息}\label{tab:multicolumn}\end{table}
	\section{限定性詞語}
		\subparagraph{必須(必要)} 所飾義項是檔案的絕對的不可或缺的條件;
		\subparagraph{禁止(不得)} 所飾義項是檔案的絕對的不可取的條件;
		\subparagraph{應當(推薦)} 存在條件忽略所飾義項。但應權衡行爲所可能造成的結果,包括完整性問題等;
		\subparagraph{不應(不推薦)} 存在條件使所飾義項可接受。但應權衡行爲所可能造成的結果;
		\subparagraph{允許(可以)} 可忽略所飾義項。表示此項僅僅是爲了拓展行爲的非必要的內容;
\end{document}
